% !TeX root = ./main.tex

%Configure document
\documentclass{article}
\title{My title}
\author{Your Name}

%Set encoding
\usepackage[T1]{fontenc}
\usepackage[utf8]{inputenc}

%Change language
\usepackage[english]{babel}

%Import general packages
\usepackage{lmodern}
\usepackage{float}
\usepackage{listings}
\usepackage{xcolor}
\usepackage{xparse}
\usepackage{blindtext}
\usepackage{bookmark}

%Import and config packages in here
%---Graphics
\usepackage{graphics}
\usepackage{graphicx}
\graphicspath{{../assets/}} %declare assets folders

%---Minted
\usepackage{minted}
\usemintedstyle{paraiso-light} %for a list of options available, in the terminal, run pygmentize -L styles

%---Watermark
\usepackage[angle=0]{draftwatermark}
\SetWatermarkText{\includegraphics[width=\textwidth]{../assets/watermark.png}} %add watermark if needed
\SetWatermarkScale{0.75}

%---Glossaries
\usepackage[acronym]{glossaries}
\makeglossaries{}
\newacronym{coc}{CoC}{Convention over Configuration}

%Begin document
%--------------------------------------
\begin{document}

\maketitle
\printacronyms{}

\section{Convention over Configuration}

\acrfull{coc} (also known as coding by convention) is a software design paradigm used by software frameworks that attempts to decrease the number of decisions that a developer using the framework is required to make without necessarily losing flexibility.

\section{Code Examples}

The following is a code example of a recursive Fibonacci method using the \textbf{minted} package.

\begin{minted}{ruby}
def fibonacci(n)
  n <= 1 ? n :  fibonacci(n - 1) + fibonacci(n - 2)
end
puts fibonacci 10
# => 55
\end{minted}

The following is an example of how to import code from a file:

\inputminted[tabsize=2]{ruby}{../assets/code/example.rb}

% generates whole document with dummy lorem ipsum text
\blindmathpaper{}

%End document
\end{document}
